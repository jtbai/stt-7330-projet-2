\section{Introduction}

L'apprentissage statistique est une discipline des l'intelligence artificielle permettant de transmettre des connaissances à une machine apprenante. La nature des connaissance transmises est très variée de telle sorte qu'il est possible d'apprendre à une machine à réaliser des tâche sur à peut près n'importe quel sujet. De plus, si ces connaissances contiennent des relations de causalité entre les différetes variables, il sera possible d'entraîner une machine à faire des généralisations sur les données. Ces généralisation peuvent, par la suite, permettre à la machine de faire des prédiction pour des nouvelles données qui n'ont jamais été 





Le but de ce document est de prédire la surface où jouent les joueurs de tenis (3 classes: Clay, Hard, carpet) et son effet sur les performances des joueurs. Pour cela, on importe la base de donnée à partir de Github "..." qui contient 50 variables environ. Dans ce qui suit, on essayera de trouver des caractéristiques qui va nous permettent d'avoir des modèles et les tendances potentiels en faisant du feature engineering. Aprés avoir obtenus nos nouvelles variables explicatives, on réduira la dimension de la base de donnée en effectuant une ACP. Premièrement, on appliquera les méthodes classiques de classification supervisée (le classifieur naif de bayes, Classifieur par arbre, classifieur régression logistique, Analyse discriminante linéaire, Analyse discriminante quadratique, Classifieur forêt aléatoire). Deuxièment, On applique les réseaux de neuronnes. Pour enfin, appliquer la méthode d'ensemble qui  utilisent les différents algorithmes d'apprentissages vus auparavant ce qui a comme but d'avoir une performance predictive plus élevés de ce qu'on pourrait avoir si on utilise une seule méthode d'apprentissage.