\section{Introduction}

L'apprentissage statistique est une discipline de l'intelligence artificielle permettant de transmettre des connaissances à une machine apprenante. La nature des connaissances transmises est très variée de telle sorte qu'il est possible d'apprendre à une machine à réaliser des tâche sur n'importe quel sujet. De plus, si ces connaissances contiennent des relations de causalité entre les différetes variables, il sera possible d'entraîner une machine à faire des généralisations sur les données. Ces généralisation peuvent, par la suite, lui permettre de faire des prédictions sur de nouvelles données qui n'ont jamais été observées.

Le projet réalisé par notre équipe porte sur l'utilisation de techniques dans le contexte d'une tâche d'apprentissage supervisé. La tâche à accomplir est de prédire, suite à un match de tennis, la surface sur laquelle ce match été joué.  Pour accomplir cette tâche, différentes techniques d'exploration de données et d'apprentissage automatique à complexité variée ont été utilisées.  Dans ce rapport, l'équipe présente les algorithmes utilisés, les résultats obtenus ainsi que leurs conclusions quant à la tâche à accomplir.

