\section{Méthodologie}

\subsection{Présentation du jeu de données}
Le jeu de données utilisé dans ce travail est distribuées par Jeff Sackmann vai github. Il comprennent différentes statistiques sur les match de officiels de l'association des professionnels de tennis depuis 1968. Le jeu de données comprend premièrement la variable que la machine apprenante devra prédire, soit la surface sur laquelle un match de tennis est joué. Cette variable peut prendre 3 valeurs soit: Clay, Hard, Carpet. Les autres variables sont utilisées comme variables explicative.






Le but de ce document est de prédire la surface où jouent les joueurs de tenis (3 classes: ) et son effet sur les performances des joueurs. Pour cela, on importe la base de donnée à partir de Github "..." qui contient 50 variables environ. Dans ce qui suit, on essayera de trouver des caractéristiques qui va nous permettent d'avoir des modèles et les tendances potentiels en faisant du feature engineering. Aprés avoir obtenus nos nouvelles variables explicatives, on réduira la dimension de la base de donnée en effectuant une ACP. Premièrement, on appliquera les méthodes classiques de classification supervisée (le classifieur naif de bayes, Classifieur par arbre, classifieur régression logistique, Analyse discriminante linéaire, Analyse discriminante quadratique, Classifieur forêt aléatoire). Deuxièment, On applique les réseaux de neuronnes. Pour enfin, appliquer la méthode d'ensemble qui  utilisent les différents algorithmes d'apprentissages vus auparavant ce qui a comme but d'avoir une performance predictive plus élevés de ce qu'on pourrait avoir si on utilise une seule méthode d'apprentissage.


\subsection{Prétraîtement des données}

\subsection{Classifieurs utilisés}

Les algorithmes d'apprentissage automatique utilisés pour le projets varient en complexité. Nous désirions tester différents niveaux de complexité afin de déterminer si une augmentation de la complexité était justifiée relativement à la qualité des prédictions. Nous avons librement sous divisés les algorithme en 3 catégories distinctes. 

La première catégorie d'algoritme les plus simple ne demande qu'une compréhension du modèle afin de l'utiliser. Les modèles comme 
- le classifieur de naïf de Bayes, 
- le classifieur régression logistique, 
- l'analyse discriminante linéaire et 
- l'analyse discriminante quadratique 

font partie de cette première catégorie. Nous considérons qu'ils sont simple car ils sont utilisables directement avec une fonction R. 

La deuxième catégorie d'algoritmes sont un peu plus complexes et demandent un ajustement de leur hyper-paramètres. Les modèles comme 
- le classifieur par arbre, 
- le classifieur forêt aléatoire et 
- le classifieur SVM 

font parti de cette deuxième catégories. La méthodologie de la sélection des hyper-paramètres est présentée dans une section ultérieure.

Finalement, la troisième catégorie d'algoritme sont les plus complexe et demandent l'utilisation de l'expérience et l'intuition du statisticien. Les modèles comme 
- les réseaux de neuronnes et 
- le modèle par emsemble 

font parti de cette troisième catégorie. L'utilisation et l'architecture de ces modèles seront présentés dans une section ultérieure

\subsection{}


\subsection{}

