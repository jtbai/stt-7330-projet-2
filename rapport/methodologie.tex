\section{Méthodologie}

\subsection{Présentation du jeu de données}
Le jeu de données utilisé dans ce travail est distribuées par Jeff Sackmann vai github. Il comprennent différentes statistiques sur les match de officiels de l'association des professionnels de tennis depuis 1968. Le jeu de données comprend premièrement la variable que la machine apprenante devra prédire, soit la surface sur laquelle un match de tennis est joué. Cette variable peut prendre 3 valeurs soit: Clay, Hard, Carpet. Les autres variables sont utilisées comme variables explicative.






Le but de ce document est de prédire la surface où jouent les joueurs de tenis (3 classes: ) et son effet sur les performances des joueurs. Pour cela, on importe la base de donnée à partir de Github "..." qui contient 50 variables environ. Dans ce qui suit, on essayera de trouver des caractéristiques qui va nous permettent d'avoir des modèles et les tendances potentiels en faisant du feature engineering. Aprés avoir obtenus nos nouvelles variables explicatives, on réduira la dimension de la base de donnée en effectuant une ACP. Premièrement, on appliquera les méthodes classiques de classification supervisée (le classifieur naif de bayes, Classifieur par arbre, classifieur régression logistique, Analyse discriminante linéaire, Analyse discriminante quadratique, Classifieur forêt aléatoire). Deuxièment, On applique les réseaux de neuronnes. Pour enfin, appliquer la méthode d'ensemble qui  utilisent les différents algorithmes d'apprentissages vus auparavant ce qui a comme but d'avoir une performance predictive plus élevés de ce qu'on pourrait avoir si on utilise une seule méthode d'apprentissage.


\subsection{Prétraîtement des données}

\subsection{Classifieurs}

\subsubsection{Classifieur de Bayes Naïf}

\subsubsection{Classifieur par arbre}

\subsubsection{Classifieur régression logistique}

\subsubsection{Analyse discriminante linéaire}

\subsubsection{Analyse discriminante quadratique}

\subsubsection{Classifieur forêt aléatoire}

\subsubsection{Classifieur SVM}

\subsubsection{Réseau de neurones}

\subsection{Sélection des hyperparamètres}

\subsection{Modèle par ensemble}

