\section{Résultats}
Maintenant que nous avons définit les différents modèles à entraîner ainsi que leurs hyperparamètres respectifs, il ne reste plus qu'à évaluer leur performance et choisir le modèle final. Une fois que nous aurons déterminé quel sera ce modèle final, il ne restera plus qu'à faire une brève analyse de celui-ci avec une dernière évaluation de sa performance sur les données de validation, qui n'ont toujours pas été vues par le modèle.  

\subsection{Évaluation des performances}
Dans le but d'orienter notre choix de modèle, nous devons définir des mesures de performances qui permettront de comparer les modèles entres eux. Étant donné la nature du problème, soit de prédire la surface d'un match de tennis selon différentes informations, il est clair que la précision est un choix évident. 

\subsection{Choix du modèle}
Note: Ici on pourrait evident comparer les performances plus haut mais aussi parler de l'interpretation et voir si le trade-off en vaut la peine ...

\subsection{Présentation du modèle final}
Note: Ici on roule le modele sur les donnees de validation (split_group = 3), on fait le lien avec le classifieur au hasard pour avoir une idée de la performance (benchmark) et on pourrait tenter de faire une genre d'interpreation (quelles vairables parlent, quelles parlent pas, ...)