

\documentclass{article}
\usepackage[utf8]{inputenc}
\usepackage[T1]{fontenc}
\usepackage{lmodern}
\usepackage[french]{babel}
\usepackage{multicol}
\usepackage{amsmath}
\usepackage{amssymb}
\usepackage{enumitem}
\numberwithin{equation}{section}
\usepackage{algorithm}
\usepackage{algorithmic}
\usepackage{graphicx}
\usepackage{float}
\usepackage{moreverb}
\usepackage{url}
\usepackage{subcaption}
\setlength\parindent{0pt}
\usepackage{bbm}

\usepackage{hyperref}
\hypersetup{
	colorlinks=true,
	linkcolor=blue,
	filecolor=magenta,      
	urlcolor=cyan,
}

\urlstyle{same}

\begin{document}
\begin{titlepage}
	\centering % Centre everything on the title page
	
	\scshape % Use small caps for all text on the title page
	
	\vspace*{\baselineskip} % White space at the top of the page
	
	%------------------------------------------------
	%	Title
	%------------------------------------------------
	
	\rule{\textwidth}{1.6pt}\vspace*{-\baselineskip}\vspace*{2pt} % Thick horizontal rule
	\rule{\textwidth}{0.4pt} % Thin horizontal rule
	
	\vspace{0.75\baselineskip} % Whitespace above the title
	
	{\LARGE Projet 2 : Classification de sols des parties de tennis\\} % Title
	\raisebox{-1pt}{\includegraphics[height=20pt, width=20pt]{GitHub-Mark-32px}}\ \href{https://github.com/stecaron/stt-7330-projet-2}{Lien GitHub}
	\vspace{0.75\baselineskip} % Whitespace below the title
	
	\rule{\textwidth}{0.4pt}\vspace*{-\baselineskip}\vspace{3.2pt} % Thin horizontal rule
	\rule{\textwidth}{1.6pt} % Thick horizontal rule
	
	\vspace{2\baselineskip} % Whitespace after the title block
	
	%------------------------------------------------
	%	Subtitle
	%------------------------------------------------
	{\scshape\Large Jean-Thomas Baillargeon \\Christopher Blier-Wong\\Stéphane Caron\\Sofia Harrouch\\} % Editor list
	Pour le cours STT-7330\\
	Méthode d'analyse des données \\% Subtitle or further description
	
	\vspace*{3\baselineskip} % Whitespace under the subtitle
	
	%------------------------------------------------
	%	Editor(s)
	%------------------------------------------------
	
	Présenté le 20 avril 2018 à la professeure
	
	\vspace{0.5\baselineskip} % Whitespace before the editors
	
	{\scshape\Large Anne-Sophie Charest \\} % Editor list
	
	\vspace{0.5\baselineskip} % Whitespace below the editor list
	
	\textit{Département de mathématiques et de statistiques\\Faculté des sciences et de génie\\Université Laval} % Editor affiliation
	
	\vfill % Whitespace between editor names and publisher logo
	
	%------------------------------------------------
	%	Publisher
	%------------------------------------------------
	
        
        \includegraphics[height=1cm, width=2.5cm]{UL_P.pdf}\\
        Faculté des sciences et de génie\\
        Université Laval\\
        Hiver 2018       
\end{titlepage}

\section{Résultats}

Vu qu'on est en présence des classifieurs à multiple classes (l=3, leur performance peut être calculer par différentes mesures,

- \textit{Average accuracy:} la moyenne de l'éfficacité par classe c'est à dire son pouvoir moyen d'accorder l'observation à la bonne classe. \\\\
\begin{center}
$\frac{\sum_{i=1}^{l} \frac{tp_i +tn_i}{tp_i + fn_i + fp_i+ tn_i}}{l}$
\end{center}

- \textit{Error rate:} la moyenne de l'erreur du classifieur (le fait de se tromper de classes) par classe.\\ \\
\begin{center}
$\frac{\sum_{i=1}^{l} \frac{fp_i +fn_i}{tp_i + fn_i + fp_i+ tn_i}}{l}$
\end{center}

- $Précision_{\mu}$ : l'accord entre les classes intiales avec celle obtenues par le classifieur. \\\\
\begin{center}
$\frac{\sum_{i=1}^{l} tp_i}{\sum_{i=1}^{l} (tp_i + fp_i)}$
\end{center}

- \textit{$Recall_{\mu}$ (sensibilité):} l'efficacité du classifieur à identifier la vrai classe.\\\\
\begin{center}
$\frac{\sum_{i=1}^{l} tp_i}{\sum_{i=1}^{l} (tp_i + fn_i)}$
\end{center}

- \textit{$F score_{\mu}$:} cette mesure est une combinaison des mesure de précision et de \textit{recall}, permet de calculer la précision du (c'est à dire combien d'observation sont bien classé) ainsi la robustesse (autrement dit combien de fois il affecte la mauvaise classe).\\\\
\begin{center}
$\frac{2 * précision_{\mu} Recall_{\mu}}{précision_{\mu} + Recall_{\mu}}$
\end{center}

- $Précision_{M}$ la moyenne de l'accord entre la classe prédit et la la classe réelle calculé pour chaque classe.\\\\
\begin{center}
$\frac{\sum_{i=1}^{l} \frac{tp_i}{(tp_i + fp_i)}}{l}$
\end{center}

- \textit{$Recall_{M}$: } la moyenne de l'efficacité du classifieur à identifier la vrai classe calculée pour chaque classe.\\
\begin{center}
$\frac{\sum_{i=1}^{l} \frac{tp_i}{(tp_i + fn_i)}}{l}$
\end{center}

- \textit{$Fscore_{M}$}: la moyenne de F score calculé pour chaque classe.\\\\
\begin{center}
$\frac{2 * Précision_{M} * Recall_{M}}{Précision_{M} + Recall_{M}}$
\end{center}


En fait, l'évaluation se fait pour chaque classe i (1, 2, 3) en calculant:  vrai positif ($tp_i$), vrai négatif ($tn_i$), faux positif ($fp_i$), faux negatif ($fn_i$), \textit{$accuracy_i$}, précision, \textit{$recall_i$}, et la qualité globale de la classification se fait en deux façon: soit en calculant la moyenne des mesures calculées  pour chaque classe i (appelée macro-moyenne avec un indice M), soit calculer la somme cumulative  des tp, fn, tn, fp de toutes les classes et après calculer les mesures de performances (ce qu'on appelle micro-moyenne $\mu$).







\end{document}

