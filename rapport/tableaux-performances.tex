\documentclass{article}

\usepackage{booktabs}
\usepackage{longtable}
\usepackage{array}
\usepackage{multirow}
\usepackage[utf8]{inputenc}
\usepackage[table]{xcolor}
\usepackage{wrapfig}
\usepackage{float}
\usepackage[french]{babel}
\usepackage{colortbl}
\usepackage{pdflscape}
\usepackage{tabu}
\usepackage{threeparttable}
\usepackage{threeparttablex}
\usepackage[normalem]{ulem}
\usepackage{makecell}

\usepackage{Sweave}
\begin{document}
\Sconcordance{concordance:tableaux-performances.tex:tableaux-performances.Rnw:%
1 19 1 1 0 2 1 1 62 1 2 23 0 1 2 23 0 1 3 1 20 16 0 1 2 4 1}



\rowcolors{2}{gray!6}{white}
\begin{table}

\caption{\label{tab:}Performances des différents modèles sur le jeu de données d'entraînement selon les différentes mesures de performance.}
\centering
\resizebox{\linewidth}{!}{\begin{tabular}[t]{lclcl}
\hiderowcolors
\toprule
Modèles & Exactitude & F-Score & Exactitude avec PCA & F-Score avec PCA\\
\midrule
\showrowcolors
Forêt aléatoire & 1.00 & 1.00 & NA & NA\\
Ensemble & 0.83 & 0.88 & 0.74 & 0.81\\
Réseau de neurones & 0.79 & 0.85 & NA & NA\\
Multinomial & 0.75 & 0.81 & 0.74 & 0.81\\
LDA & 0.75 & 0.81 & 0.74 & 0.81\\
\addlinespace
Arbre de décision & 0.74 & 0.81 & 0.73 & 0.80\\
QDA & 0.72 & 0.79 & 0.73 & 0.80\\
Bayes & 0.70 & 0.78 & 0.73 & 0.80\\
SVM Gaussien & 0.69 & 0.77 & NA & NA\\
\bottomrule
\end{tabular}}
\end{table}
\rowcolors{2}{white}{white}\rowcolors{2}{gray!6}{white}
\begin{table}

\caption{\label{tab:}Performances des différents modèles sur le jeu de données d'entraînement selon les différentes mesures de performance.}
\centering
\resizebox{\linewidth}{!}{\begin{tabular}[t]{lclcl}
\hiderowcolors
\toprule
Modèles & Exactitude & F-Score & Exactitude avec PCA & F-Score avec PCA\\
\midrule
\showrowcolors
Réseau de neurones & 0.78 & 0.84 & NA & NA\\
Ensemble & 0.76 & 0.82 & 0.75 & 0.81\\
Multinomial & 0.76 & 0.82 & 0.75 & 0.81\\
LDA & 0.76 & 0.82 & 0.75 & 0.81\\
Forêt aléatoire & 0.75 & 0.82 & NA & NA\\
\addlinespace
Arbre de décision & 0.75 & 0.81 & 0.74 & 0.80\\
QDA & 0.72 & 0.79 & 0.74 & 0.80\\
Bayes & 0.71 & 0.78 & 0.74 & 0.81\\
SVM Gaussien & 0.70 & 0.78 & NA & NA\\
\bottomrule
\end{tabular}}
\end{table}
\rowcolors{2}{white}{white}
\rowcolors{2}{gray!6}{white}
\begin{table}

\caption{\label{tab:}Matrice de confusion pour le modèle final.}
\centering
\begin{tabular}[t]{llcc}
\hiderowcolors
\toprule
  & Hard & Clay & Grass\\
\midrule
\showrowcolors
Hard & 16566 & 5468 & 2738\\
Clay & 1389 & 5375 & 173\\
Grass & 298 & 138 & 740\\
\bottomrule
\end{tabular}
\end{table}
\rowcolors{2}{white}{white}



\end{document}
